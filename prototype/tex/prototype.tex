\documentclass{beamer}

\usepackage{minted}

\title{Design Patterns}
\subtitle{Prototype}
\author{Heerdyes Mahapatro}

\usetheme{Berkeley}
\usecolortheme{dove}
\begin{document}
	\frame {
		\titlepage
	}
	\frame {
		\frametitle{Prototype}
		Intent - Need to decide the class of an object dynamically from a menu of classes.
		Motivation - create an object from a dynamic list of classes.
		\begin{itemize}
		    \item useful for static languages like C++
		    \item Smalltalk, Objective C and Python already have the concept of prototype (class object)
		    \item In Self and JavaScript all instantiation happens through prototyping
		\end{itemize}
	}
	\frame{
	    \frametitle{Prototype - contd}
	    \begin{itemize}
	        \item Structure \cite{EGamma98}
    	    \item \includegraphics[width=0.6\paperwidth]{prototype.jpg}
    	\end{itemize}
	}
	\begin{frame}[fragile]
	    \frametitle{Prototype - multiple classes}
	    \rule{\textwidth}{1pt}
	    \scriptsize
	    \begin{minted}[fontsize=\tiny]{python}
class SoftSynth:
    def __init__(self,polyphony,channels,latency):
        self.polyphony=polyphony
        self.channels=channels
        self.latency=latency
        
    def __str__(self):
        return 'SoftSynth[poly=%d,ch=%d,lat=%d]'%(self.polyphony,self.channels,self.latency)

class AnalogSynth:
    def __init__(self,polyphony):
        self.polyphony=polyphony
    
    def __str__(self):
        return 'AnalogSynth[poly=%d]'%self.polyphony

class SynthWorkstation:
    def __init__(self,polyphony,channels):
        self.polyphony=polyphony
        self.channels=channels
        
    def __str__(self):
        return 'SynthWorkstation[poly=%d,ch=%d]'%(self.polyphony,self.channels)
	    \end{minted}
	    \rule{\textwidth}{1pt}
	\end{frame}
	\begin{frame}[fragile]
	    \frametitle{Prototype - multiple classes contd}
	    \rule{\textwidth}{1pt}
	    \scriptsize
	    \begin{minted}[fontsize=\tiny]{python}
instmenu={
    'sw': (SynthWorkstation, 128, 16),
    'as': (AnalogSynth, 4),
    'ss': (SoftSynth, 256, 256, 50)
}

print('enter inst code [sw,as,ss]: ')
inst=input()
syn=instmenu[inst]
# reminiscent of lisp's car and cdr
i0=syn[0](*syn[1:])
print(i0)
	    \end{minted}
	    \rule{\textwidth}{1pt}
	\end{frame}
	\begin{frame}[fragile]
	    \frametitle{Prototype - CLI shell}
	    \rule{\textwidth}{1pt}
	    \scriptsize
	    \begin{minted}[fontsize=\tiny]{python}
import copy

class BigBangPrototype:
    def __init__(self,attrs):
        self.attrs=attrs
    
    def clone(self):
        return BigBangPrototype(copy.deepcopy(self.attrs))
        
    def __str__(self):
        return 'attrs: '+str(self.attrs)


# original prototype that is the source of all objects
selobj='O'
symtab={
    'O':BigBangPrototype({})
}
while True:
    print('1. show objects')
    print('2. create object')
    print('3. customize object')
    print('q. quit')
    ch=input()
	    \end{minted}
	    \rule{\textwidth}{1pt}
	\end{frame}
	\begin{frame}[fragile]
	    \frametitle{Prototype - CLI shell contd}
	    \rule{\textwidth}{1pt}
	    \scriptsize
	    \begin{minted}[fontsize=\tiny]{python}
    if ch=='1':
        print('-----')
        for k in symtab.keys():
            print(k,symtab[k])
        print('-----')
    elif ch=='2':
        print('[mkobj] select prototyping source: ',end='')
        src=input().strip()
        print('[mkobj] object name: ',end='')
        onm=input().strip()
        symtab[onm]=symtab[src].clone()
    elif ch=='3':
        print('[custom] object name: ',end='')
        selobj=input().strip()
        custobj=symtab[selobj]
        while True:
            print('[customizer] > ',end='')
            inarr=input().strip().split(' ')
            if inarr[0]=='_q':
                break
            custobj.attrs[inarr[0]]=inarr[1]
        symtab[selobj]=custobj
    elif ch=='q':
        break
    else:
        print('unknown option: '+ch)
	    \end{minted}
	    \rule{\textwidth}{1pt}
	\end{frame}
	\frame{
	    \frametitle{Prototype - Remarks}
	    \begin{itemize}
	        \item Unnecessary in many cases for languages like python, smalltalk and ruby \cite{pyproto}
	        \item First-class functions and classes in python make the semantics very convenient
	    \end{itemize}
	}
	\frame{
	    \frametitle{References}
	    \bibliographystyle{amsalpha}
	    \bibliography{sources.bib}
	}

\end{document}
